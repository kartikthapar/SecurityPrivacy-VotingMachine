\documentclass[]{article}

\usepackage{setspace}
\usepackage{ctable}
\usepackage{multirow}
\usepackage{mdwlist}
\onehalfspacing
\newcommand{\indentitem}{\setlength\itemindent{25pt}}

\begin{document}

\begin{center}
	\textsc{\LARGE CS 600.443 --- Security \& Privacy}\\[0.5cm]
	\textsc{\Large Project 2 --- Initial Brief}\\[0.25cm]
	\line(1,0){320}\\[15pt]
\end{center}

%----------------------------------------------------------------------------

\textsc{\huge Team Members}\\

Following are the team members ---

\begin{enumerate*}
	{\indentitem \item Parsia Hakimian}
	{\indentitem \item Kartik Thapar}
\end{enumerate*}

It is understood that given the opportunity, a team of three people would be a better fit for the scope of \textbf{Projects 2 \& 3}.\\[0.5cm]

\textsc{\huge Project Outline}\\

The group discussed the platform and the scope of web applications. Web applications seem to have strong advantages over native C++/Assembly code along with its disadvantages. One of the prime advantages was in implementing a platform independent solution with tremendous ease of voting. Unfortunately for our purposes the advantages are irrelevant as the other team will need to audit and process our code. Doing so would require the testing group to setup a good size of external resources.

Byte code platforms are usually present in all systems but the voting program will inherit any and every weakness and the developers of the e-voting program will not be able to account for \emph{0-day exploits} on the hosting platform.

The group discussed in detail, the implementation and acceptance of a number of programming languages aligned with the notion of availability of prevailing code for voting systems. These languages include--- C, C++, Perl, Python, Java, Objective-C, etc. It is understood that Objective-C provides one of the best options to implement object oriented code with well established cryptographic libraries but given the availability of open-source and public code for voting machine systems, a mixture of Perl, Python and C is chosen to implement such a system.\\

\textsc{\Large Progress}\\

The first week has been more of a brainstorming session where the group has stressed on researching relevant background theory on voting machines and voting practices in the United States. Relevant background theory includes--- number of votes cast in presidential elections, number and names of states involved in presidential elections, types of voting machines, design and implementation of software in embedded voting machines, scope of voting machines, technical problems and security issues with specific voting machines, law and policy associated with ballot casting and elections.

The group has researched surveys and articles delimiting the security and compliance in such systems. Effort has been made to identify (and understand part of) prevailing code in compromised voting machine and other similar systems.

As described above, the group insists on the availability of one more group member to successfully accomplish tasks in Projects 2 \& 3. The group assumes that even if the two group members double their efforts, it's a case of competing two perspectives against four sets of eyes and minds in order to devise and/or find installed backdoors.

\end{document}